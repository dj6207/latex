%%%%%%%%%%%%%%%%%
% This is an sample CV template created using altacv.cls
% (v1.1.2, 1 February 2017) written by LianTze Lim (liantze@gmail.com). Now compiles with pdfLaTeX, XeLaTeX and LuaLaTeX.
%
%% It may be distributed and/or modified under the
%% conditions of the LaTeX Project Public License, either version 1.3
%% of this license or (at your option) any later version.
%% The latest version of this license is in
%%    http://www.latex-project.org/lppl.txt
%% and version 1.3 or later is part of all distributions of LaTeX
%% version 2003/12/01 or later.
%%%%%%%%%%%%%%%%

%% If you need to pass whatever options to xcolor
\PassOptionsToPackage{dvipsnames}{xcolor}

%% If you are using \orcid or academicons
%% icons, make sure you have the academicons
%% option here, and compile with XeLaTeX
%% or LuaLaTeX.
% \documentclass[10pt,a4paper,academicons]{altacv}

%% Use the "normalphoto" option if you want a normal photo instead of cropped to a circle
% \documentclass[10pt,a4paper,normalphoto]{altacv}

\documentclass[10pt,letter]{altacv}

%% AltaCV uses the fontawesome and academicon fonts
%% and packages.
%% See texdoc.net/pkg/fontawecome and http://texdoc.net/pkg/academicons for full list of symbols.
%%
%% Compile with LuaLaTeX for best results. If you
%% want to use XeLaTeX, you may need to install
%% Academicons.ttf in your operating system's font
%% folder.


% Change the page layout if you need to
\geometry{left=1cm,right=10cm,marginparwidth=8.25cm,marginparsep=.75cm,top=0.5cm,bottom=1cm}

% Change the font if you want to.

% If using pdflatex:
\usepackage[utf8]{inputenc}
\usepackage[T1]{fontenc}
\usepackage[default]{lato}
\usepackage[none]{hyphenat}
\usepackage[document]{ragged2e}

% If using xelatex or lualatex:
% \setmainfont{Lato}

% Change the colours if you want to
\definecolor{LightGreen}{HTML}{90EE90}
\definecolor{AccentGreen}{HTML}{339966}
\definecolor{AcePurple}{HTML}{800080}
\definecolor{DarkerPurple}{HTML}{5b005b}
\definecolor{SlateGrey}{HTML}{2E2E2E}
\definecolor{LightGrey}{HTML}{666666}
\definecolor{SomeBlue}{HTML}{4286f4}
% \colorlet{heading}{AcePurple}%Sepia}
% \colorlet{heading}{LightGreen}
\colorlet{heading}{AcePurple}
% \colorlet{accent}{DarkerPurple}%AccentGreen}
% \colorlet{accent}{AccentGreen}
\colorlet{accent}{DarkerPurple}
\colorlet{emphasis}{SlateGrey}
\colorlet{body}{LightGrey}

% Change the bullets for itemize and rating marker
% for \cvskill if you want to
\renewcommand{\itemmarker}{{\small\textbullet}}
\renewcommand{\ratingmarker}{\faCircle}

\begin{document}
\name{Devin Jiang}
\tagline{Seeking a Summer or Fall 2023 Software Engineering \\ Co-op/Intership}
\personalinfo{%
  % Not all of these are required!
  % You can add your own with \printinfo{symbol}{detail}
  \email{devinjiang7@gmail.com}\\\smallskip
  \github{github.com/dj6207}\\\smallskip
  % fix my github bruh
  % \homepage{}\\\smallskip
  % \phone{(917) 530-8909}\\\smallskip
  \linkedin{in/jiangdevin}\\\smallskip
  %% You MUST add the academicons option to \documentclass, then compile with LuaLaTeX or XeLaTeX, if you want to use \orcid or other academicons commands.
%   \orcid{orcid.org/0000-0000-0000-0000}
}

%% Make the header extend all the way to the right, if you want.
\begin{fullwidth}
\marginpar{\makesidebarheader% \medskip
% \bigskip
% \bigskip
% \medskip
\vspace*{6pt}

\cvsection{Education}
\cvevent{Rochester Institute of Technology}{Computer Engineering B.S. GPA: 3.7}{Expected Graduation May 2026}{}

\medskip

\cvsection{Relevant Courses}
\cvtag{Computer Science I}
\cvtag{Computer Science II}
\cvtag{Intro to Software Engineering}
\cvtag{Digital System Design I}
\cvtag{Circuits I}
\cvtag{Digital System Design II}
\cvtag{Circuits II}
\cvtag{Assembly and Embedded Programing}

\medskip

\cvsection{Skills}
\cvsubsection{Languages}
\cvtag{Python}
\cvtag{TypeScript}
\cvtag{Rust}
\cvtag{C\#}
\cvtag{Java}
\cvtag{Java Script}
\cvtag{HTML/CSS}
\cvtag{SQL}
\cvtag{VHDL}
\cvtag{ARM Assembly}
\cvtag{LaTex}

% Python,C#,Java,SQL,Assembly,VHDL,LaTex,Java Script,TypeScript,HTML,CSS

\medskip

\cvsubsection{Tools}
\cvtag{Visual Studio/Visual Studio Code}
\cvtag{Docker}
\cvtag{SQL Server Management Studio}
\cvtag{SQLite}
\cvtag{PostgreSQL}
\cvtag{Git/GitHub}
\cvtag{Intellij IDEA}
\cvtag{Linux}
\cvtag{Raspberry Pi}

%% Yeah I didn't spend too much time making all the
%% spacing consistent... sorry. Use \smallskip, \medskip,
%% \bigskip, \vpsace etc to make ajustments.

\medskip

\cvsection{Activities}

\activity{\textbf{Computer Science House}}{August 2021 - Present | Active Member}
\begin{itemize}
  \item Participated in weekly house meetings and technical seminars to learn new skills and concepts.
  \item Worked on many personal projects with guidance from technically experienced members.
\end{itemize}

\divider

\activity{\textbf{Society of Asian Scientists and Engineers}}{August 2021 - Present | Active Member}
\begin{itemize}
  \item Attended career development activities such as resume workshops and networking events.
  \item Gained career guidance from experienced members through mentorship programs.
\end{itemize}

%%%%%%%%%%%%%%%%%
% \cvsection{Activities}

% \activity{\textbf{Computer Science House}}{August 2021 - Present | Active Member}
% \begin{itemize}
%   \item Participated in weekly house meetings and technical seminars to learn new skills and concepts.
%   \item Worked on many personal projects with guidance from technically experienced members.
%   % \item Living and learning community focused on learning new skills and concepts.
%   % \item Collaborative environment where members are highly encouraged to work on personal projects.
% \end{itemize}

% \normalactivity{Computer Science House}{August 2021 - Present | Active Member}
% \begin{itemize}
%   \item Living and learning community focused on learning new skills and concepts.
%   \item Collaborative environment where members are highly encouraged to work on personal projects.
% \end{itemize}

% \activity{\textbf{Society of Asian Scientists and Engineers}}{August 2021 - Present | Active Member}
% \begin{itemize}
%   \item Attended career development activities such as resume workshops and networking events.
%   \item Gained career guidance from experienced members through mentorship programs.
%   % \item An organization based around mentorship for career development.
% \end{itemize}

% \normalactivity{Society of Asian Scientists and Engineers}{August 2021 - Present | Active Member}
% \begin{itemize}
%     \item An organization based around mentorship for career development.
% \end{itemize}
%%%%%%%%%%%%%%%%%
}
    \vspace*{-1\baselineskip}
\makecvheader
\end{fullwidth}
%% Provide the file name containing the sidebar contents as an optional parameter to \cvsection.
%% You can always just use \marginpar{...} if you do
%% not need to align the top of the contents to any
%% \cvsection title in the "main" bar.

\cvsection{Projects}

\project{Note Taking Program}{https://github.com/dj6207/notes}
% add a read me for this project
\begin{itemize}
  \item Wrote a Python program to record and send audio to a server, then transcribes the audio into text
  \item Utilized Facebook AI's Wav2vec2 automatic speech recognitioin model to transcribe audio to text
  \item Dockerized the Python program for easy deployment on various systems
% \item Wrote a Python scraper to check the official dashboard on an interval and update a SQLite database with the numbers from the official dashboard, then expose the saved data with a JSON API
% \item Consumed data from the scraper in a React app that provides all the information from the official dashboard as well as changes over a user-selected time period and graphs of all historical data
% \item Used by roughly 10\% of the student body within 24 hours of launch
\end{itemize}
\textit{\textbf{Tools:} Python, Docker}

\divider

\normalproject{Home Server}
\begin{itemize}
  \item Deployed a Network Attached Storage server using Unraid software
  \item Hosted virtual machines and docker container applications for various users  
  \item Initialized a VPN tunnel to the server using WireGuard for easy access
\end{itemize}
\textit{\textbf{Tools:} Unraid, WireGuard, Docker}

\divider

\project{Weather App}{https://github.com/dj6207/Weather}
\begin{itemize}
  \item Implemented a JavaScript website to display the current weather in various cities
  \item Utilized the OpenWeather API to obtain weather statistics 
\end{itemize}
\textit{\textbf{Tools:} JavaScript, HTML/CSS}

\divider

\project{Rush Hour Game}{https://github.com/dj6207/RushHour}
\begin{itemize}
% \item TALK more about experiences and projcet
  \item Implemented Rush Hour Gaming using the Model View Controller framework in Java
  \item Utilized the JavaFX library to create a user interface for the game
  \item Used Breadth First Search algorithm to find the shortest solution
\end{itemize}
\textit{\textbf{Tools:} Java, JavaFX}

\smallskip

\cvsection{Activities}

\normalactivity{Computer Science House}{August 2021 - Present | Active Member}
\begin{itemize}
  \item Living and learning community focused on learning new skills and concepts
  \item Collaborative environment where members are highly encouraged to work on personal projects
\end{itemize}

% \cvevent{Computer Science House}{Active Member}{August 2021 - Present}{}
% \begin{itemize}
%     \item Living and learning community focused on learning new skills and concepts
%     \item Collaborative environment where members are highly encouraged to work on personal projects
% \end{itemize}

\divider

\normalactivity{Society of Asian Scientists and Engineers}{August 2021 - Present | Active Member}
\begin{itemize}
    \item An organization based around mentorship for career development
\end{itemize}

% \cvevent{Society of Asian Scientists and Engineers}{Active Member}{August 2021 - Present}{}
% \begin{itemize}
%     \item An organization based around mentorship for career development
% \end{itemize}

\clearpage

\end{document}
